\documentclass{article}\usepackage[]{graphicx}\usepackage[]{xcolor}
% maxwidth is the original width if it is less than linewidth
% otherwise use linewidth (to make sure the graphics do not exceed the margin)
\makeatletter
\def\maxwidth{ %
  \ifdim\Gin@nat@width>\linewidth
    \linewidth
  \else
    \Gin@nat@width
  \fi
}
\makeatother

\definecolor{fgcolor}{rgb}{0.345, 0.345, 0.345}
\newcommand{\hlnum}[1]{\textcolor[rgb]{0.686,0.059,0.569}{#1}}%
\newcommand{\hlstr}[1]{\textcolor[rgb]{0.192,0.494,0.8}{#1}}%
\newcommand{\hlcom}[1]{\textcolor[rgb]{0.678,0.584,0.686}{\textit{#1}}}%
\newcommand{\hlopt}[1]{\textcolor[rgb]{0,0,0}{#1}}%
\newcommand{\hlstd}[1]{\textcolor[rgb]{0.345,0.345,0.345}{#1}}%
\newcommand{\hlkwa}[1]{\textcolor[rgb]{0.161,0.373,0.58}{\textbf{#1}}}%
\newcommand{\hlkwb}[1]{\textcolor[rgb]{0.69,0.353,0.396}{#1}}%
\newcommand{\hlkwc}[1]{\textcolor[rgb]{0.333,0.667,0.333}{#1}}%
\newcommand{\hlkwd}[1]{\textcolor[rgb]{0.737,0.353,0.396}{\textbf{#1}}}%
\let\hlipl\hlkwb

\usepackage{framed}
\makeatletter
\newenvironment{kframe}{%
 \def\at@end@of@kframe{}%
 \ifinner\ifhmode%
  \def\at@end@of@kframe{\end{minipage}}%
  \begin{minipage}{\columnwidth}%
 \fi\fi%
 \def\FrameCommand##1{\hskip\@totalleftmargin \hskip-\fboxsep
 \colorbox{shadecolor}{##1}\hskip-\fboxsep
     % There is no \\@totalrightmargin, so:
     \hskip-\linewidth \hskip-\@totalleftmargin \hskip\columnwidth}%
 \MakeFramed {\advance\hsize-\width
   \@totalleftmargin\z@ \linewidth\hsize
   \@setminipage}}%
 {\par\unskip\endMakeFramed%
 \at@end@of@kframe}
\makeatother

\definecolor{shadecolor}{rgb}{.97, .97, .97}
\definecolor{messagecolor}{rgb}{0, 0, 0}
\definecolor{warningcolor}{rgb}{1, 0, 1}
\definecolor{errorcolor}{rgb}{1, 0, 0}
\newenvironment{knitrout}{}{} % an empty environment to be redefined in TeX

\usepackage{alltt}
\usepackage[sc]{mathpazo}
\renewcommand{\sfdefault}{lmss}
\renewcommand{\ttdefault}{lmtt}
\usepackage[T1]{fontenc}
\usepackage{geometry}
\geometry{verbose,tmargin=2.5cm,bmargin=2.5cm,lmargin=2.5cm,rmargin=2.5cm}
\setcounter{secnumdepth}{2}
\setcounter{tocdepth}{2}
\usepackage[unicode=true,pdfusetitle,
 bookmarks=true,bookmarksnumbered=true,bookmarksopen=true,bookmarksopenlevel=2,
 breaklinks=false,pdfborder={0 0 1},backref=false,colorlinks=false]
 {hyperref}
\hypersetup{
 pdfstartview={XYZ null null 1}}

\makeatletter
%%%%%%%%%%%%%%%%%%%%%%%%%%%%%% User specified LaTeX commands.
\renewcommand{\textfraction}{0.05}
\renewcommand{\topfraction}{0.8}
\renewcommand{\bottomfraction}{0.8}
\renewcommand{\floatpagefraction}{0.75}

\makeatother
\IfFileExists{upquote.sty}{\usepackage{upquote}}{}
\begin{document}








The results below are generated from an R script.

\begin{knitrout}
\definecolor{shadecolor}{rgb}{0.969, 0.969, 0.969}\color{fgcolor}\begin{kframe}
\begin{alltt}
\hlcom{# •	Cuantos campos y observaciones tiene el dataframe.  Utilizar “head” y “dim”.}
\hlkwd{head}\hlstd{(airquality)} \hlcom{# -> Hay 6 campos: Ozone Solar.R Wind Temp Month Day.}
\end{alltt}
\begin{verbatim}
##      Ozone  Solar.R Wind Temp Month Day
## 1 41.00000 190.0000  7.4   67     5   1
## 2 36.00000 118.0000  8.0   72     5   2
## 3 12.00000 149.0000 12.6   74     5   3
## 4 18.00000 313.0000 11.5   62     5   4
## 5 59.11538 181.2963 14.3   56     5   5
## 6 28.00000 181.2963 14.9   66     5   6
\end{verbatim}
\begin{alltt}
\hlkwd{dim}\hlstd{(airquality)}  \hlcom{# -> 153 observaciones con 6 campos.}
\end{alltt}
\begin{verbatim}
## [1] 153   6
\end{verbatim}
\begin{alltt}
\hlcom{# •	Evaluar el dataframe con la instrucción “summary”.}
\hlcom{#     o	¿Tiene observaciones con elementos nulos (NA)?}
\hlcom{#     o	¿A que meses corresponden las observaciones?}
\hlkwd{summary}\hlstd{(airquality)}
\end{alltt}
\begin{verbatim}
##      Ozone           Solar.R           Wind             Temp           Month      
##  Min.   :  1.00   Min.   :  7.0   Min.   : 1.700   Min.   :56.00   Min.   :5.000  
##  1st Qu.: 21.00   1st Qu.:118.5   1st Qu.: 7.400   1st Qu.:72.00   1st Qu.:6.000  
##  Median : 45.00   Median :199.0   Median : 9.700   Median :79.00   Median :7.000  
##  Mean   : 46.24   Mean   :185.8   Mean   : 9.958   Mean   :77.88   Mean   :6.993  
##  3rd Qu.: 59.12   3rd Qu.:257.5   3rd Qu.:11.500   3rd Qu.:85.00   3rd Qu.:8.000  
##  Max.   :168.00   Max.   :334.0   Max.   :20.700   Max.   :97.00   Max.   :9.000  
##                   NA's   :3                                                       
##       Day      
##  Min.   : 1.0  
##  1st Qu.: 8.0  
##  Median :16.0  
##  Mean   :15.8  
##  3rd Qu.:23.0  
##  Max.   :31.0  
## 
\end{verbatim}
\begin{alltt}
\hlcom{# Hay valores NA en Ozone (37) y en Solar Radiation (7).}
\hlcom{# Los meses durante los que se realizaron las observaciones son del 5 al 9 (es decir de mayo a septiembre).}

\hlcom{# •	Temperatura máxima del viento en el mes de mayo.}
\hlkwd{max}\hlstd{(airquality[airquality}\hlopt{$}\hlstd{Month} \hlopt{==} \hlnum{5}\hlstd{,]}\hlopt{$}\hlstd{Temp)} \hlcom{# <- 81}
\end{alltt}
\begin{verbatim}
## [1] 81
\end{verbatim}
\begin{alltt}
\hlcom{# •	Media del ozono en el mes de Julio. }

\hlkwd{mean}\hlstd{(airquality[airquality}\hlopt{$}\hlstd{Month} \hlopt{==} \hlnum{7}\hlstd{,]}\hlopt{$}\hlstd{Ozone,}\hlkwc{na.rm} \hlstd{=} \hlnum{TRUE}\hlstd{)} \hlcom{# -> 59.11538}
\end{alltt}
\begin{verbatim}
## [1] 59.11538
\end{verbatim}
\begin{alltt}
\hlstd{media}\hlkwb{=}\hlkwd{mean}\hlstd{(airquality[airquality}\hlopt{$}\hlstd{Month} \hlopt{==} \hlnum{7}\hlstd{,]}\hlopt{$}\hlstd{Ozone,}\hlkwc{na.rm} \hlstd{=} \hlnum{TRUE}\hlstd{)} \hlcom{# -> 59.11538}

\hlcom{# Transformar al valor de la media los NA.}
\hlstd{airquality}\hlopt{$}\hlstd{Ozone[}\hlkwd{is.na}\hlstd{(airquality}\hlopt{$}\hlstd{Ozone)]} \hlkwb{<-} \hlstd{media}
\hlcom{# Estudiar el efecto de esta asignación sobre la desviación típica}

\hlkwd{mean}\hlstd{(airquality[airquality}\hlopt{$}\hlstd{Month} \hlopt{==} \hlnum{7}\hlstd{,]}\hlopt{$}\hlstd{Ozone)}
\end{alltt}
\begin{verbatim}
## [1] 59.11538
\end{verbatim}
\begin{alltt}
\hlcom{# •	Mes donde la temperatura fue mayor.}
\hlstd{airquality[}\hlkwd{max}\hlstd{(airquality}\hlopt{$}\hlstd{Temp),]}\hlopt{$}\hlstd{Month} \hlcom{# -> Agosto (8)}
\end{alltt}
\begin{verbatim}
## [1] 8
\end{verbatim}
\begin{alltt}
\hlcom{# •	Mes donde la temperatura y el ozono fue mayor.}
\hlkwd{length}\hlstd{(airquality[airquality}\hlopt{$}\hlstd{Temp} \hlopt{>} \hlnum{90} \hlopt{&} \hlstd{airquality}\hlopt{$}\hlstd{Ozone} \hlopt{<} \hlnum{100}\hlstd{,}\hlstr{"Month"}\hlstd{])} \hlcom{# -> 13}
\end{alltt}
\begin{verbatim}
## [1] 13
\end{verbatim}
\begin{alltt}
\hlcom{# •	Haciendo un estudio de los datos, ¿Qué podemos concluir? }
\hlcom{# ¿Existe alguna relación entre las variables Ozono, Temperatura y Radiación Solar?}
\hlcom{# Se recomienda hacer la media mes a mes de cada variable. }

\hlkwd{mean}\hlstd{(airquality}\hlopt{$}\hlstd{Ozone[airquality}\hlopt{$}\hlstd{Month} \hlopt{==} \hlnum{5}\hlstd{])}
\end{alltt}
\begin{verbatim}
## [1] 29.34119
\end{verbatim}
\begin{alltt}
\hlkwd{mean}\hlstd{(airquality}\hlopt{$}\hlstd{Ozone[airquality}\hlopt{$}\hlstd{Month} \hlopt{==} \hlnum{6}\hlstd{])}
\end{alltt}
\begin{verbatim}
## [1] 50.2141
\end{verbatim}
\begin{alltt}
\hlkwd{mean}\hlstd{(airquality}\hlopt{$}\hlstd{Ozone[airquality}\hlopt{$}\hlstd{Month} \hlopt{==} \hlnum{7}\hlstd{])}
\end{alltt}
\begin{verbatim}
## [1] 59.11538
\end{verbatim}
\begin{alltt}
\hlkwd{mean}\hlstd{(airquality}\hlopt{$}\hlstd{Ozone[airquality}\hlopt{$}\hlstd{Month} \hlopt{==} \hlnum{8}\hlstd{])}
\end{alltt}
\begin{verbatim}
## [1] 59.82506
\end{verbatim}
\begin{alltt}
\hlkwd{mean}\hlstd{(airquality}\hlopt{$}\hlstd{Ozone[airquality}\hlopt{$}\hlstd{Month} \hlopt{==} \hlnum{9}\hlstd{])}
\end{alltt}
\begin{verbatim}
## [1] 32.37051
\end{verbatim}
\begin{alltt}
\hlkwd{mean}\hlstd{(airquality}\hlopt{$}\hlstd{Temp[airquality}\hlopt{$}\hlstd{Month} \hlopt{==} \hlnum{5}\hlstd{])}
\end{alltt}
\begin{verbatim}
## [1] 65.54839
\end{verbatim}
\begin{alltt}
\hlkwd{mean}\hlstd{(airquality}\hlopt{$}\hlstd{Temp[airquality}\hlopt{$}\hlstd{Month} \hlopt{==} \hlnum{6}\hlstd{])}
\end{alltt}
\begin{verbatim}
## [1] 79.1
\end{verbatim}
\begin{alltt}
\hlkwd{mean}\hlstd{(airquality}\hlopt{$}\hlstd{Temp[airquality}\hlopt{$}\hlstd{Month} \hlopt{==} \hlnum{7}\hlstd{])}
\end{alltt}
\begin{verbatim}
## [1] 83.90323
\end{verbatim}
\begin{alltt}
\hlkwd{mean}\hlstd{(airquality}\hlopt{$}\hlstd{Temp[airquality}\hlopt{$}\hlstd{Month} \hlopt{==} \hlnum{8}\hlstd{])}
\end{alltt}
\begin{verbatim}
## [1] 83.96774
\end{verbatim}
\begin{alltt}
\hlkwd{mean}\hlstd{(airquality}\hlopt{$}\hlstd{Temp[airquality}\hlopt{$}\hlstd{Month} \hlopt{==} \hlnum{9}\hlstd{])}
\end{alltt}
\begin{verbatim}
## [1] 76.9
\end{verbatim}
\begin{alltt}
\hlstd{media_Solar.R_5}\hlkwb{=}\hlkwd{mean}\hlstd{(airquality}\hlopt{$}\hlstd{Solar.R[airquality}\hlopt{$}\hlstd{Month} \hlopt{==} \hlnum{5}\hlstd{],}\hlkwc{na.rm} \hlstd{=} \hlnum{TRUE}\hlstd{)}
\hlkwd{sqrt}\hlstd{(}\hlkwd{var}\hlstd{(airquality}\hlopt{$}\hlstd{Solar.R[airquality}\hlopt{$}\hlstd{Month} \hlopt{==} \hlnum{5}\hlstd{],}\hlkwc{na.rm}\hlstd{=}\hlnum{TRUE}\hlstd{))}
\end{alltt}
\begin{verbatim}
## [1] 107.1295
\end{verbatim}
\begin{alltt}
\hlstd{media_Solar.R_6}\hlkwb{=}\hlkwd{mean}\hlstd{(airquality}\hlopt{$}\hlstd{Solar.R[airquality}\hlopt{$}\hlstd{Month} \hlopt{==} \hlnum{6}\hlstd{],}\hlkwc{na.rm} \hlstd{=} \hlnum{TRUE}\hlstd{)}
\hlstd{media_Solar.R_7}\hlkwb{=}\hlkwd{mean}\hlstd{(airquality}\hlopt{$}\hlstd{Solar.R[airquality}\hlopt{$}\hlstd{Month} \hlopt{==} \hlnum{7}\hlstd{],}\hlkwc{na.rm} \hlstd{=} \hlnum{TRUE}\hlstd{)}
\hlstd{media_Solar.R_8}\hlkwb{=}\hlkwd{mean}\hlstd{(airquality}\hlopt{$}\hlstd{Solar.R[airquality}\hlopt{$}\hlstd{Month} \hlopt{==} \hlnum{8}\hlstd{],}\hlkwc{na.rm} \hlstd{=} \hlnum{TRUE}\hlstd{)}
\hlstd{media_Solar.R_9}\hlkwb{=}\hlkwd{mean}\hlstd{(airquality}\hlopt{$}\hlstd{Solar.R[airquality}\hlopt{$}\hlstd{Month} \hlopt{==} \hlnum{9}\hlstd{],}\hlkwc{na.rm} \hlstd{=} \hlnum{TRUE}\hlstd{)}

\hlcom{# Transformar los NA.}
\hlkwd{table}\hlstd{(}\hlkwd{is.na}\hlstd{(airquality}\hlopt{$}\hlstd{Solar.R))}
\end{alltt}
\begin{verbatim}
## 
## FALSE  TRUE 
##   150     3
\end{verbatim}
\begin{alltt}
\hlstd{indices5}\hlkwb{=}\hlkwd{which}\hlstd{(}\hlkwd{is.na}\hlstd{(airquality}\hlopt{$}\hlstd{Solar.R[airquality}\hlopt{$}\hlstd{Month} \hlopt{==} \hlnum{5}\hlstd{]))}
\hlcom{# ojo, estamos cambiando todos los datos sin haber salvado la anterior versión}
\hlcom{# del dataframe}
\hlstd{airquality}\hlopt{$}\hlstd{Solar.R[airquality}\hlopt{$}\hlstd{Month} \hlopt{==} \hlnum{5}\hlstd{][indices5]}\hlkwb{=}\hlstd{media_Solar.R_5}
\hlkwd{mean}\hlstd{(airquality}\hlopt{$}\hlstd{Solar.R[airquality}\hlopt{$}\hlstd{Month} \hlopt{==} \hlnum{5}\hlstd{])}
\end{alltt}
\begin{verbatim}
## [1] 181.2963
\end{verbatim}
\begin{alltt}
\hlkwd{sqrt}\hlstd{(}\hlkwd{var}\hlstd{(airquality}\hlopt{$}\hlstd{Solar.R[airquality}\hlopt{$}\hlstd{Month} \hlopt{==} \hlnum{5}\hlstd{]))}
\end{alltt}
\begin{verbatim}
## [1] 107.1295
\end{verbatim}
\begin{alltt}
\hlcom{# Si se hace la media del ozono, temperatura y radiacion solar podemos observar como más o menos todos incrementan a la par.}
\hlcom{# El mes de junio es el único que presenta algo de variación.}
\hlcom{# Se podría concluir que todas las variables indicadas tiene relación entre ellas.}
\end{alltt}
\end{kframe}
\end{knitrout}

The R session information (including the OS info, R version and all
packages used):

\begin{knitrout}
\definecolor{shadecolor}{rgb}{0.969, 0.969, 0.969}\color{fgcolor}\begin{kframe}
\begin{alltt}
\hlkwd{sessionInfo}\hlstd{()}
\end{alltt}
\begin{verbatim}
## R version 4.3.1 (2023-06-16)
## Platform: x86_64-pc-linux-gnu (64-bit)
## Running under: Ubuntu 20.04.6 LTS
## 
## Matrix products: default
## BLAS:   /usr/lib/x86_64-linux-gnu/atlas/libblas.so.3.10.3 
## LAPACK: /usr/lib/x86_64-linux-gnu/atlas/liblapack.so.3.10.3;  LAPACK version 3.9.0
## 
## locale:
##  [1] LC_CTYPE=es_ES.UTF-8       LC_NUMERIC=C               LC_TIME=es_ES.UTF-8       
##  [4] LC_COLLATE=es_ES.UTF-8     LC_MONETARY=es_ES.UTF-8    LC_MESSAGES=es_ES.UTF-8   
##  [7] LC_PAPER=es_ES.UTF-8       LC_NAME=C                  LC_ADDRESS=C              
## [10] LC_TELEPHONE=C             LC_MEASUREMENT=es_ES.UTF-8 LC_IDENTIFICATION=C       
## 
## time zone: Europe/Madrid
## tzcode source: system (glibc)
## 
## attached base packages:
## [1] stats     graphics  grDevices utils     datasets  methods   base     
## 
## other attached packages:
## [1] knitr_1.44       factoextra_1.0.7 ggplot2_3.4.3    arules_1.7-6     Matrix_1.6-1.1  
## 
## loaded via a namespace (and not attached):
##  [1] gtable_0.3.4         xfun_0.40            recipes_1.0.8        ggrepel_0.9.3       
##  [5] lattice_0.21-9       vctrs_0.6.3          tools_4.3.1          generics_0.1.3      
##  [9] stats4_4.3.1         parallel_4.3.1       tibble_3.2.1         fansi_1.0.5         
## [13] highr_0.10           pkgconfig_2.0.3      ModelMetrics_1.2.2.2 data.table_1.14.8   
## [17] lifecycle_1.0.3      farver_2.1.1         compiler_4.3.1       stringr_1.5.0       
## [21] munsell_0.5.0        codetools_0.2-19     DALEX_2.4.3          htmltools_0.5.6.1   
## [25] class_7.3-22         yaml_2.3.7           prodlim_2023.08.28   pillar_1.9.0        
## [29] MASS_7.3-60          gower_1.0.1          iterators_1.0.14     rpart_4.1.19        
## [33] foreach_1.5.2        nlme_3.1-163         parallelly_1.36.0    lava_1.7.2.1        
## [37] tidyselect_1.2.0     digest_0.6.33        stringi_1.7.12       future_1.33.0       
## [41] dplyr_1.1.3          reshape2_1.4.4       purrr_1.0.2          listenv_0.9.0       
## [45] labeling_0.4.3       splines_4.3.1        cowplot_1.1.1        fastmap_1.1.1       
## [49] grid_4.3.1           colorspace_2.1-0     cli_3.6.1            magrittr_2.0.3      
## [53] survival_3.5-7       utf8_1.2.3           future.apply_1.11.0  withr_2.5.1         
## [57] scales_1.2.1         xgboost_1.7.5.1      lubridate_1.9.3      timechange_0.2.0    
## [61] rmarkdown_2.25       globals_0.16.2       nnet_7.3-19          timeDate_4022.108   
## [65] evaluate_0.22        hardhat_1.3.0        caret_6.0-94         rlang_1.1.1         
## [69] Rcpp_1.0.11          glue_1.6.2           pROC_1.18.4          ipred_0.9-14        
## [73] rstudioapi_0.15.0    jsonlite_1.8.7       R6_2.5.1             plyr_1.8.9
\end{verbatim}
\begin{alltt}
\hlkwd{Sys.time}\hlstd{()}
\end{alltt}
\begin{verbatim}
## [1] "2023-10-31 22:29:26 CET"
\end{verbatim}
\end{kframe}
\end{knitrout}


\end{document}

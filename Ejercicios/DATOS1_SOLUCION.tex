\documentclass{article}\usepackage[]{graphicx}\usepackage[]{xcolor}
% maxwidth is the original width if it is less than linewidth
% otherwise use linewidth (to make sure the graphics do not exceed the margin)
\makeatletter
\def\maxwidth{ %
  \ifdim\Gin@nat@width>\linewidth
    \linewidth
  \else
    \Gin@nat@width
  \fi
}
\makeatother

\definecolor{fgcolor}{rgb}{0.345, 0.345, 0.345}
\newcommand{\hlnum}[1]{\textcolor[rgb]{0.686,0.059,0.569}{#1}}%
\newcommand{\hlstr}[1]{\textcolor[rgb]{0.192,0.494,0.8}{#1}}%
\newcommand{\hlcom}[1]{\textcolor[rgb]{0.678,0.584,0.686}{\textit{#1}}}%
\newcommand{\hlopt}[1]{\textcolor[rgb]{0,0,0}{#1}}%
\newcommand{\hlstd}[1]{\textcolor[rgb]{0.345,0.345,0.345}{#1}}%
\newcommand{\hlkwa}[1]{\textcolor[rgb]{0.161,0.373,0.58}{\textbf{#1}}}%
\newcommand{\hlkwb}[1]{\textcolor[rgb]{0.69,0.353,0.396}{#1}}%
\newcommand{\hlkwc}[1]{\textcolor[rgb]{0.333,0.667,0.333}{#1}}%
\newcommand{\hlkwd}[1]{\textcolor[rgb]{0.737,0.353,0.396}{\textbf{#1}}}%
\let\hlipl\hlkwb

\usepackage{framed}
\makeatletter
\newenvironment{kframe}{%
 \def\at@end@of@kframe{}%
 \ifinner\ifhmode%
  \def\at@end@of@kframe{\end{minipage}}%
  \begin{minipage}{\columnwidth}%
 \fi\fi%
 \def\FrameCommand##1{\hskip\@totalleftmargin \hskip-\fboxsep
 \colorbox{shadecolor}{##1}\hskip-\fboxsep
     % There is no \\@totalrightmargin, so:
     \hskip-\linewidth \hskip-\@totalleftmargin \hskip\columnwidth}%
 \MakeFramed {\advance\hsize-\width
   \@totalleftmargin\z@ \linewidth\hsize
   \@setminipage}}%
 {\par\unskip\endMakeFramed%
 \at@end@of@kframe}
\makeatother

\definecolor{shadecolor}{rgb}{.97, .97, .97}
\definecolor{messagecolor}{rgb}{0, 0, 0}
\definecolor{warningcolor}{rgb}{1, 0, 1}
\definecolor{errorcolor}{rgb}{1, 0, 0}
\newenvironment{knitrout}{}{} % an empty environment to be redefined in TeX

\usepackage{alltt}
\usepackage[sc]{mathpazo}
\renewcommand{\sfdefault}{lmss}
\renewcommand{\ttdefault}{lmtt}
\usepackage[T1]{fontenc}
\usepackage{geometry}
\geometry{verbose,tmargin=2.5cm,bmargin=2.5cm,lmargin=2.5cm,rmargin=2.5cm}
\setcounter{secnumdepth}{2}
\setcounter{tocdepth}{2}
\usepackage[unicode=true,pdfusetitle,
 bookmarks=true,bookmarksnumbered=true,bookmarksopen=true,bookmarksopenlevel=2,
 breaklinks=false,pdfborder={0 0 1},backref=false,colorlinks=false]
 {hyperref}
\hypersetup{
 pdfstartview={XYZ null null 1}}

\makeatletter
%%%%%%%%%%%%%%%%%%%%%%%%%%%%%% User specified LaTeX commands.
\renewcommand{\textfraction}{0.05}
\renewcommand{\topfraction}{0.8}
\renewcommand{\bottomfraction}{0.8}
\renewcommand{\floatpagefraction}{0.75}

\makeatother
\IfFileExists{upquote.sty}{\usepackage{upquote}}{}
\begin{document}



\title{\title{\title{}}}



\maketitle
The results below are generated from an R script.

\begin{knitrout}
\definecolor{shadecolor}{rgb}{0.969, 0.969, 0.969}\color{fgcolor}\begin{kframe}
\begin{alltt}
\hlcom{# Librería para leer XLSX}
\hlkwd{library}\hlstd{(readxl)}
\hlkwd{library}\hlstd{(dplyr)}

\hlcom{# leemos los datos previamente salvados}
\hlstd{df} \hlkwb{<-} \hlkwd{read_excel}\hlstd{(}\hlstr{"Dry_Bean_Dataset.xlsx"}\hlstd{)}

\hlcom{# Número de observaciones y variables}
\hlkwd{dim}\hlstd{(df)}
\end{alltt}
\begin{verbatim}
## [1] 13611    17
\end{verbatim}
\begin{alltt}
\hlcom{# Particiones}

\hlcom{# mediante una semilla conseguimos que el ejercicio sea reproducible}
\hlkwd{set.seed}\hlstd{(}\hlnum{12321}\hlstd{)}

\hlcom{# creamos índices}
\hlstd{ntotal} \hlkwb{<-} \hlkwd{dim}\hlstd{(df)[}\hlnum{1}\hlstd{]}
\hlstd{indices} \hlkwb{<-} \hlnum{1}\hlopt{:}\hlstd{ntotal}
\hlstd{ntrain} \hlkwb{<-} \hlstd{ntotal} \hlopt{*} \hlnum{.6}
\hlstd{ntest} \hlkwb{<-} \hlstd{ntotal} \hlopt{*}\hlnum{.2}
\hlstd{indices.train} \hlkwb{<-} \hlkwd{sample}\hlstd{(indices, ntrain,} \hlkwc{replace} \hlstd{=} \hlnum{FALSE}\hlstd{)}
\hlstd{indices.test}  \hlkwb{<-} \hlkwd{sample}\hlstd{(indices[}\hlopt{-}\hlstd{indices.train],ntest,}\hlkwc{replace}\hlstd{=}\hlnum{FALSE}\hlstd{)}
\hlstd{indices.valid} \hlkwb{<-} \hlstd{indices[}\hlopt{-}\hlkwd{c}\hlstd{(indices.train,indices.test)]}

\hlcom{# Usamos el 60% de la base de datos como conjunto de entrenamiento, 20% como conjunto de test y 20% como validación}
\hlstd{train}  \hlkwb{<-} \hlstd{df[indices.train, ]}
\hlstd{test}   \hlkwb{<-} \hlstd{df[indices.test, ]}
\hlstd{valid}  \hlkwb{<-} \hlstd{df[indices.valid,]}

\hlkwd{dim}\hlstd{(train)}
\end{alltt}
\begin{verbatim}
## [1] 8166   17
\end{verbatim}
\begin{alltt}
\hlkwd{dim}\hlstd{(test)}
\end{alltt}
\begin{verbatim}
## [1] 2722   17
\end{verbatim}
\begin{alltt}
\hlkwd{dim}\hlstd{(valid)}
\end{alltt}
\begin{verbatim}
## [1] 2723   17
\end{verbatim}
\begin{alltt}
\hlcom{# Media de la variable AREA}
\hlstd{media.train}\hlkwb{=}\hlkwd{mean}\hlstd{(train}\hlopt{$}\hlstd{Area)}
\hlstd{sd.train}\hlkwb{=}\hlkwd{sqrt}\hlstd{(}\hlkwd{var}\hlstd{(train}\hlopt{$}\hlstd{Area))}
\hlkwd{mean}\hlstd{(test}\hlopt{$}\hlstd{Area)}
\end{alltt}
\begin{verbatim}
## [1] 52377.31
\end{verbatim}
\begin{alltt}
\hlkwd{mean}\hlstd{(valid}\hlopt{$}\hlstd{Area)}
\end{alltt}
\begin{verbatim}
## [1] 53244.92
\end{verbatim}
\begin{alltt}
\hlcom{# Escalamos la variable.}

\hlstd{train}\hlkwb{=}
  \hlstd{train} \hlopt
  \hlkwd{mutate}\hlstd{(}\hlkwc{Area_Scale}\hlstd{=}\hlkwd{scale}\hlstd{(Area))}

\hlkwd{mean}\hlstd{(train}\hlopt{$}\hlstd{Area_Scale)}
\end{alltt}
\begin{verbatim}
## [1] 1.944833e-17
\end{verbatim}
\begin{alltt}
\hlkwd{var}\hlstd{(train}\hlopt{$}\hlstd{Area_Scale)}
\end{alltt}
\begin{verbatim}
##      [,1]
## [1,]    1
\end{verbatim}
\begin{alltt}
\hlcom{# Aplicamos la misma transformación en los datos de test y validación}

\hlstd{test}\hlkwb{=}
  \hlstd{test} \hlopt
  \hlkwd{mutate}\hlstd{(}\hlkwc{Area_Scale}\hlstd{=(Area}\hlopt{-}\hlstd{media.train)}\hlopt{/}\hlstd{sd.train)}

\hlkwd{mean}\hlstd{(test}\hlopt{$}\hlstd{Area_Scale)}
\end{alltt}
\begin{verbatim}
## [1] -0.02785735
\end{verbatim}
\begin{alltt}
\hlkwd{var}\hlstd{(test}\hlopt{$}\hlstd{Area_Scale)}
\end{alltt}
\begin{verbatim}
## [1] 0.8295407
\end{verbatim}
\begin{alltt}
\hlstd{valid}\hlkwb{=}
  \hlstd{valid} \hlopt
  \hlkwd{mutate}\hlstd{(}\hlkwc{Area_Scale}\hlstd{=(Area}\hlopt{-}\hlstd{media.train)}\hlopt{/}\hlstd{sd.train)}

\hlkwd{mean}\hlstd{(valid}\hlopt{$}\hlstd{Area_Scale)}
\end{alltt}
\begin{verbatim}
## [1] 0.001295361
\end{verbatim}
\begin{alltt}
\hlkwd{var}\hlstd{(valid}\hlopt{$}\hlstd{Area_Scale)}
\end{alltt}
\begin{verbatim}
## [1] 1.024713
\end{verbatim}
\end{kframe}
\end{knitrout}

The R session information (including the OS info, R version and all
packages used):

\begin{knitrout}
\definecolor{shadecolor}{rgb}{0.969, 0.969, 0.969}\color{fgcolor}\begin{kframe}
\begin{alltt}
\hlkwd{sessionInfo}\hlstd{()}
\end{alltt}
\begin{verbatim}
## R version 4.3.1 (2023-06-16)
## Platform: x86_64-pc-linux-gnu (64-bit)
## Running under: Ubuntu 20.04.6 LTS
## 
## Matrix products: default
## BLAS:   /usr/lib/x86_64-linux-gnu/atlas/libblas.so.3.10.3 
## LAPACK: /usr/lib/x86_64-linux-gnu/atlas/liblapack.so.3.10.3;  LAPACK version 3.9.0
## 
## locale:
##  [1] LC_CTYPE=es_ES.UTF-8       LC_NUMERIC=C               LC_TIME=es_ES.UTF-8       
##  [4] LC_COLLATE=es_ES.UTF-8     LC_MONETARY=es_ES.UTF-8    LC_MESSAGES=es_ES.UTF-8   
##  [7] LC_PAPER=es_ES.UTF-8       LC_NAME=C                  LC_ADDRESS=C              
## [10] LC_TELEPHONE=C             LC_MEASUREMENT=es_ES.UTF-8 LC_IDENTIFICATION=C       
## 
## time zone: Europe/Madrid
## tzcode source: system (glibc)
## 
## attached base packages:
## [1] stats     graphics  grDevices utils     datasets  methods   base     
## 
## other attached packages:
## [1] readxl_1.4.3     caret_6.0-94     lattice_0.21-9   ggplot2_3.4.3    rpart.plot_3.1.1
## [6] rpart_4.1.19     caTools_1.18.2   dplyr_1.1.3      ISLR2_1.3-2     
## 
## loaded via a namespace (and not attached):
##  [1] gtable_0.3.4         xfun_0.40            recipes_1.0.8        tzdb_0.4.0          
##  [5] vctrs_0.6.3          tools_4.3.1          bitops_1.0-7         generics_0.1.3      
##  [9] stats4_4.3.1         parallel_4.3.1       proxy_0.4-27         tibble_3.2.1        
## [13] fansi_1.0.5          highr_0.10           ModelMetrics_1.2.2.2 pkgconfig_2.0.3     
## [17] Matrix_1.6-1.1       data.table_1.14.8    lifecycle_1.0.3      stringr_1.5.0       
## [21] compiler_4.3.1       farver_2.1.1         tinytex_0.47         munsell_0.5.0       
## [25] codetools_0.2-19     htmltools_0.5.6.1    class_7.3-22         yaml_2.3.7          
## [29] prodlim_2023.08.28   pillar_1.9.0         MASS_7.3-60          gower_1.0.1         
## [33] iterators_1.0.14     foreach_1.5.2        nlme_3.1-163         parallelly_1.36.0   
## [37] lava_1.7.2.1         tidyselect_1.2.0     digest_0.6.33        stringi_1.7.12      
## [41] future_1.33.0        reshape2_1.4.4       purrr_1.0.2          listenv_0.9.0       
## [45] labeling_0.4.3       splines_4.3.1        fastmap_1.1.1        grid_4.3.1          
## [49] colorspace_2.1-0     cli_3.6.1            magrittr_2.0.3       survival_3.5-7      
## [53] utf8_1.2.3           e1071_1.7-13         future.apply_1.11.0  readr_2.1.4         
## [57] withr_2.5.1          scales_1.2.1         lubridate_1.9.3      timechange_0.2.0    
## [61] rmarkdown_2.25       globals_0.16.2       nnet_7.3-19          timeDate_4022.108   
## [65] cellranger_1.1.0     hms_1.1.3            evaluate_0.22        knitr_1.44          
## [69] hardhat_1.3.0        rlang_1.1.1          Rcpp_1.0.11          glue_1.6.2          
## [73] pROC_1.18.4          ipred_0.9-14         rstudioapi_0.15.0    R6_2.5.1            
## [77] plyr_1.8.9
\end{verbatim}
\begin{alltt}
\hlkwd{Sys.time}\hlstd{()}
\end{alltt}
\begin{verbatim}
## [1] "2023-11-01 20:39:27 CET"
\end{verbatim}
\end{kframe}
\end{knitrout}


\end{document}
